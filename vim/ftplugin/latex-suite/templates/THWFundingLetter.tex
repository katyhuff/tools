\documentclass{letter}
\usepackage{graphicx}
%\address{Kathryn Huff \\ 1500 Engineering Research Building \\ Room 434 \\ Madison, WI 53706}
\signature{ Kathryn Huff \\ Treasurer of The Hacker Within}
\begin{document}

%
% Hi, this is a template for producing a document requesting funding 
% from companies who might be interested in funding the hacker within.
% You're welcome to change my name to yours, completely rewrite this 
% document, and/or send it to anyone and everyone. 
% Please let me know, of course, if you do get any responses, since I 
% can try to send a tax-ID number to the company so they might write 
% off the contribution on their taxes.
% Thanks!
%


 \begin{letter}{RecipientName\\
CompanyName\\
123 Company Drive\\
City, State 99999
}
\begin{center}
\includegraphics[height=1cm]{thwlogo-medium.eps}
\end{center}
	\opening{Dear RecipientName,}

At the University of Wisconsin, a new student organization has emerged. The Hacker Within is a scientific computing skill-sharing group based in the Engineering Physics Department but open to anyone. Our mission is to cultivate more productive and efficient researchers by providing a forum for sharing computing tools and best practices.

The Hacker Within was formed last year with the hope to address an unmet need for software carpentry skills in the university setting. With hard work and boundless enthusiasm, our peer-taught workshops and biweekly skill-sharing meetings have garnered university recognition and a diverse, dedicated member base. Our talented, motivated membership finds a sense of community in gathering together to trade tips and tricks as well as to ask questions and grapple with problems encountered in our programming endeavors. Finally, last year a few members of The Hacker Within contributed hundreds of volunteer-hours to conduct two multiple day workshops; one on UNIX computing and the other on the C++ programming language. These workshops met with high attendance, engaged our peers, introduced skills to a broad audience, and improved our own teaching and programming proficiency.

I am emailing today because The Hacker Within is a new organization with big plans and greatly needs the initial support of companies like CompanyName, who shares our goals for technical education and would benefit from contributing to our mission. We would be very grateful for your support and recognize that without your financial backing, the goals we seek to accomplish would be unreachable. Our executive committee has planned several activities for this coming year that promise to engage computationally curious minds from talented undergraduate and graduate students of many disciplines to faculty and the broader Madison community.

We would appreciate consideration of the following proposal to help defray the cost for three areas of aspiration: 1) sponsoring part of The Hacker Within Python Bootcamp 2) contributing to student's technical growth by assisting in sending them to a Python conference, and 3) providing support for biweekly skill sharing sessions by sponsoring coffee and snacks for 16 meetings during the school year.  The following sections provide a detailed description of the activities that we are asking you to consider supporting.

\textbf{Peer Education – Python Bootcamp January 11-15, 2010}

THW would be thankful if CompanyName could sponsor refreshments for our upcoming Python Bootcamp in January of 2010. Encouraged by the grateful reception of our past workshops, ten students are volunteering to organize and teach this workshop on the Python programming language. Taught over three days for a total of twelve hours, PyBc2010 intends to transform students with little or no Python programming skills into fully functional Python programmers through lectures and hands on programming activities. In support of the bootcamp, we would like to request that CompanyName assist in the purchase of snacks and/or sandwiches. This is our most desperately needed resource. Charitable funding institutions are rarely able to donate food-items, but a time-intensive event such as this is much more successful with available refreshments. 

We expect two snack breaks and one lunch break per day and are expecting approximately 60 attendees. We have asked a the student government to sponsor our webhosting, facility costs, and tee-shirts for \$1000, approximately half of our predicted budget. We would be most grateful if you would be willing to sponsor the other half (\$1000) in support of refreshments. In return, we can offer that CompanyName would be prominently advertised as a sponsor in the flyers, emails and advertisements preceding the bootcamp as well as during the bootcamp in the program, on the presentations, and in the introductory and closing remarks. If requested, CompanyName could send a representative or materials advertising CompanyName. On an introductory slide, there could even be information about future internships, employment, current news, general facts about CompanyName, or any other information that CompanyName would want to share with computationally inclined students and faculty.

\textbf{Technical Development – Assisting THW Members to Attend PyCon2010}

Conferences are an essential opportunity for our members to learn about the broader computational community and to develop connections with other students, leading researchers, and companies in the field. The students who attend PyCon2010 will bring new Python tools back to The Hacker Within's biweekly meetings and contribute to a permanent body of peer-teaching material that we host at hackerwithin.org. This opportunity will allow THW members to work on their presentation skills, participate in tutorial presentations, and network with professionals in the worldwide Open Source community. We anticipate 5 students to attend this year’s PyCon2010 in Atlanta, Georgia from February 17th to 21st.  In addition to applying for waived registration fees from the conference organizers, we have received university funding for five flights (\$325 from NWA) and one conference hotel room for four nights(\$159/night). We would be grateful if you would consider supplementing that funding with your financial support of one flight and one more hotel room for a total of \$961.




\textbf{Community Building –  Biweekly Skill Sharing Sessions}

Our biweekly meetings are vital to the mission of The Hacker Within. Every other Friday afternoon from 3:15 to 4:00 pm, THW members gather in a small conference room on the engineering campus to hear a talk about some computational tool and discuss questions and discoveries in their programming lives. A variety of students attend in order to learn and teach in a comfortable setting, and we find that snacks are an essential ice-breaker. For this reason, our biweekly meetings are not only a significant part of our mission, but a significant cost as well. Our fearless leader, Milad Fatenejad, has provided brownies for nearly every meeting since the inception of The Hacker Within. THW is asking leaders in the computational field and applying to the Wisconsin Experience Foundation to help us pull together enough capital to fund the years' biweekly snacks. During the weeks that you choose to fund our snacks, CompanyName would have the chance to do some advertising in the verbal announcements and on the presentations which are posted online. The cost of this activity would be \$15 per week. We expect to have 16 more meetings this academic year, for a grand total of \$240 a year.

The total to sponsor three THW activities for this academic year is \$2,201. We appreciate your consideration of this proposal and hope that you will support our organization.
If you have any questions, please feel free to call me at (608) 263-1709 or email at khuff@cae.wisc.edu.

	\closing{Sincerely,}

 \end{letter}
\end{document}

